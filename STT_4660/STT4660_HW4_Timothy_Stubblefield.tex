\documentclass{article}
\usepackage{amsfonts}
\usepackage{enumitem}
\usepackage{graphicx}
\usepackage{amssymb}

\title{STT 4660 \\ Homework \#4}
\author{Timothy Stubblefield}
\date{\today}

\usepackage[svgnames]{xcolor}
\usepackage{listings}

\lstset{language=R,
	basicstyle=\small\ttfamily,
	stringstyle=\color{DarkGreen},
	otherkeywords={0,1,2,3,4,5,6,7,8,9},
	morekeywords={TRUE,FALSE},
	deletekeywords={data,frame,length,as,character},
	keywordstyle=\color{blue},
	commentstyle=\color{DarkGreen},
}

\begin{document}
	
\maketitle

\section*{2.15 (b)}
	\begin{enumerate}[label = \arabic*)]
		\item Refer to the Airfreght Breakage dataset from Problem 1.21
		\item Given a new shipment of two transfers, we want to find 99 percent confidence interval for this new shipment.
		\item Since we have two transfers, we know that,
		\[x_{new} = 2\]
		\item And from Problem 1.21, we know that the linear regression model is,
		\[\hat{y} = 10.2 +4x\]
		\item Thus, we can solve for $\hat{y}_{new}$ with the following formula,
		\[\hat{y}_{new} = 10.2 + 4x_{new}\]
		\item Substituting in the value for $x_{new}$, we get,
		\[\hat{y}_{new} = 10.2 + 4(2)\]
		\[\hat{y}_{new} = 10.2 + 8\]
		\[\hat{y}_{new} = 18.2\]
		\item Now, use the following formula to construct a 99 percent confidence interval for $y_{new}$,
		\[\hat{y}_{new} \pm t_{\alpha/2} * S_{pred}\] 
		\item First, we need to solve for $S_{pred}$, with the following formula,
		\[S_{pred}^2 = \hat{\sigma}^2[1 + \frac{1}{n} + \frac{(x_{new}-\bar{x})^2}{S_{xx}}]\]
		\item From the previous results, we know that,
		\[\hat{\sigma}^2 = MSE = 2.2\]
		\item From the same problem, we also know that,
		\[S_{xx} = \sum(x_i - \bar{x})^2 = 10\]
		And, that $n = 10 \qquad \qquad \bar{x} = 1$
		\item Substituting in these values, we can calculate $S_{pred}^2$ as such,
		\[S_{pred}^2 = 2.2[1+ \frac{1}{10} + \frac{(2-1)^2}{10}]\]
		\[S_{pred}^2 = 2.2[1 + \frac{1}{10} + \frac{1}{10}]\]
		\[S_{pred}^2 = 2.2[\frac{5}{5} + \frac{1}{5}]\]
		\[S_{pred}^2 = 2.2(\frac{6}{5})\]
		\[S_{pred}^2 = 2.64\]
		\item Now, compute $S_{pred}$ by taking square root of $S_{pred}^2$,
		\[S_{pred} = \sqrt{2.64}\]
		\[S_{pred} = 1.624807681\]
		\item Since we want a 99 percent confidence interval, we use $\alpha = 0.01$.
		\item Now we need to compute the t-value, that is,
		\[t_{\alpha/2}(n-2) = t_{0.01/2}(10-2)\]
		\[t_{0.005}(8) = 3.355387\]
		\item Now, we can construct the 99 percent confidence interval for $y_{new}$ by substituting in the appropriate values,
		\[18.2 \pm 3.355387 * 1.624807681\]
		\[18.2 \pm  5.774355273\]
		\item Thus, the 99 percent confidence interval for $y_{new}$ is,
		\[(12.42564473, 23.97435527)\]
		\item Therefore, we are 99 percent confident that $y_{new}$ lies in the range of (12.42564473, 23.97435527).
	\end{enumerate}

\section*{2.17}
	\begin{enumerate}[label = \arabic*)]
		\item Given the F-test of
		\[H_0: \beta_1 = 0\]
		\[H_a: \beta_1 \neq 0\]
		The analyst concluded that the p-value = 0.033.
		\item For the F-test, the analyst concluded $H_a: \beta_1 \neq 0$.
		\item Since $H_a$ was concluded, that means that $H_0$ was rejected. Therefore, that means p-value = 0.033 $< \alpha$.
		\item Thus, the the $\alpha$ is greater than 0.033.
		\item Now, if the analyst used an $\alpha = 0.01$, then,
		\[p-value = 0.033 > \alpha = 0.01\]
		\item Therefore, with $\alpha = 0.01$, the analyst would not reject $H_0$ and would thus have evidence that $\beta_1 = 0$.
	\end{enumerate}

\section*{2.25}
	\begin{enumerate}[label = \alph*)]
		\item Referring to the Airfreight Breakage data of Problem 1.21, we want to set up the ANOVA table.
		\begin{enumerate}[label = \arabic*)]
			\item Here is the ANOVA Table from SAS
			\centering
			\includegraphics[width=10cm]{C:/Users/timst/R/Practice/STT 4660/ANOVA_Table_2_25.PNG}
		\end{enumerate}
		\item Now, we want to construct an F-test with $\alpha = 0.05$.
		\begin{enumerate}[label = \arabic*)]
			\item For the F-test, we are testing the hypotheses,
			\[H_0: \beta_1 = 0\]
			\[H_a: \beta_1 \neq 0\]
			\item Well, we new that,
			\[F_{obs} = \frac{MSR}{MSE} = \frac{160}{2.2}\]
			\[F_{obs} = 72.727272\]
			\item Use the p-value method to find,
			\[p-value = P(F > 72.727272)\]
			\item From R, we determined that p-value = 0.00002784
			\item Thus, we have the result,
			\[p-value = 0.00002784 < \alpha = 0.05\]
			\item Therefore, reject $H_0$ and conclude $H_a$.
			\item Thus, we can say that there is evidence to show that $\beta_1 \neq 0$ and subsequently, that there is a linear relationship between X and Y.
		\end{enumerate}
		\item Determine the t* statistic and show that it is numerically equivalent to the F-test.
		\begin{enumerate}[label = \arabic*)]
			\item Let's perform a T-test on $\beta$ using the following formula,
			\[t_{obs} = \frac{b_1}{S_{b_1}}\]
			\item From Problem 2.6 in HW3, we know $S_{b_1} = 0.469041576$
			\item Substituting in appropriate values, we get,
			\[t_{obs} = \frac{4}{0.469041576}\]
			\[t_{obs} = 8.528028654\]
			\item Now compute the p-value for the t-test using R or SAS, and we get,
			\[p-value = 0.00002784\]
			\item Notice, the p-value for F-test and the T-test are the same. Thus, the two tests are equivalent.
		\end{enumerate}
		\item Finally, lets calculatte $R^2$ and r.
		\begin{enumerate}
			\item To calculate the coefficient of determination, $R^2$, use the forumula,
			\[R^2 = \frac{SSR}{SSTO}\]
			\item Filling in the values from the ANOVA Table, we have,
			\[R^2 = \frac{160}{177.6}\]
			\[R^2 = 0.900900900\]
			\item Therefore, around 90 percent of the variation in Y is explained by introducing X into the regression model.
		\end{enumerate}
	\end{enumerate}

\section*{2.30}
	\begin{enumerate}[label = \alph*)]
		\item Using the Crime Rate Data from 1.28, use a t-test with $\alpha = 0.01$ to determine if there is a linear relationship between crime rate and percentage of high school graduates.
		\begin{enumerate}[label = \arabic*)]
			\item Since we are testing to see if X and Y are linearly related, the hypothesis situation will be,
			\[H_0: \beta_1 = 0\]
			\[H_a: \beta_1 \neq 0\]
			\item From the SAS output, the linear regression function is,
			\[\hat{y} = 20518 - 170.57519x\]
			\item Here is the scatterplot of the data,
				\centering
				\includegraphics[width=10cm]{C:/Users/timst/R/Practice/STT 4660/Plot_for_2.30.png}
				
			\item Now, we need to calculate the $t_{obs}$ using the formula,
			\[t_{obs} = \frac{b_1 - \beta_{10}}{S_{b_1}}\]
			\item So, we know from the SAS output that,
			\[b_1 = -170.57519, \quad \qquad \beta_{10} = 0, \quad \qquad S_{b_1} = 41.57433\]
			\item Substituting in these values, we get,
			\[t_{obs} = \frac{-170.57519 - 0}{41.57433}\]
			\[t_{obs} = \frac{-170.57519}{41.57433}\]
			\[t_{obs} = -4.102896908\]
			\item From the notes, we know that 
			\[t = \frac{b_1}{S_{b_1}} \thicksim t(n-2) \qquad  under \qquad H_0: \beta_1 = 0\]
			\item Now, use R or SAS to compute the p-value, that is,
			\[p-value = 0.001202528\]
			\item So, compare p-value to $\alpha$, as such,
			\[p-value = 0.001202528 < 0.01 = \alpha\]
			\item Since p-value $< \alpha$, then we reject $H_0$ and conclude $H_a$.
			\item Therefore, we have evidence that shows $\beta_1 \neq 0$, and subsequently, that there is a linear relationship between high school graduation rate and crime rate.
		\end{enumerate}
		\item Now, lets's estimate $\beta_1$ with a 99 percent confidence interval.
		\begin{enumerate}[label = \arabic*)]
			\item A 99 percent confidence interval for $\beta_1$ is,
			\[b_1 \pm t_{\alpha/2}(n-2) * S_{b_1}\]
			And, $\alpha = 0.01$, so $\quad \frac{\alpha}{2} = 0.005$
			\item First, we need to compute $t_{\alpha/2}(n-2)$, as such,
			\[t_{\alpha/2}(n-2) = t_{0.005}(84-2)\]
			\item Using a two-tailed T-test, the evaluated t-value from R is,
			\[t_{0.005}(82) = 2.637123\]
			\item Now, from previous results we know that $b_1 = -170.57519$ and $S_{b_1} = 41.57433$.
			\item Thus, substitute in the values to the equation from (1), to find the confidence interval,
			\[-170.57519 \pm 2.637123 * 41.57433\]
			\[-170.57519 \pm 109.6366219\]
			\item Thus, the 99 percent confidence interval for $\beta_1$ is,
			\[(-280.2118119, -60.9385681)\]
			\item Therefore, we are 99 percent confident that the true value of $\beta_1$ lies in the range of (-280.2118119, -60.9385681).
		\end{enumerate}
	\end{enumerate}

\section*{2.32}
	\begin{enumerate}[label = \alph*)]
		\item Referring to the Crime Rate data use in 2.30, we want to find the full and reduced regression models.\\
		Here is the ANOVA Table for Crime Rate Data using SAS
		
			\includegraphics[width=10cm]{C:/Users/timst/R/Practice/STT 4660/ANOVA_Table_2_32.PNG}
		\begin{enumerate}[label = \arabic*)]
			\item From Problem 2.30, we determined the regression model to be 
			\[\hat{y} = 20518 - 170.57519x\]
			\item Since we are using the Simple Linear Regression (SLR) model, the full model is the regression function with $\beta_1 \neq 0$ and the reduced model is the intercept only model of the regression function where $\beta_1 = 0$.
			\item Thus, the full and reduced models are as follows,
			\[Full Model:\quad \hat{y} = 20518 - 170.57519x\]
			\[Reduced Model:\quad \hat{y} = 20518 \]
		\end{enumerate}
		\item Now, let's obtain some useful statistics.
		\begin{enumerate}[label = \arabic*)]
			\item As detailed in the notes, the $SSE(F)$ uses the full model so,
			\[SSE(F) = SSE\]
			 Looking at the ANOVA Table from (a), we have,
			\[SSE(F) = 5552112\]
			\item Now, we want to find $SSE(R)$.
			 The notes specify that in the SLR model,
			\[SSE(R) = \sum (y_i - \hat{y}_i)^2 = \sum (y_i - \bar{y})^2 = SSTO\] 
			From the ANOVA Table, we can see that,
			\[SSE(R) = SSTO = 548736108\]
			\item Next, we need to calculate $df_F$
			From the book, we can see that 
			\[df_F = n-2\]
			Also, we know n =84.
			Thus, substituting in values, we get,
			\[df_F = 84-2\]
			\[df_F = 82\]
			\item Now, let's calculae $df_R$
			From the book, we can determine that
			\[df_R = n-1\]
			Thus, substituting in the value, n=84, we get,
			\[df_R = 84-1\]
			\[df_R = 83\]
			\item Find the statistic F* for the general linear test.
			The formula for F* for the general linear test is,
			\[F* = \frac{SSLF/1}{SSE(F)/(n-2)}\]
			First, we need to compute the SSLF, using the following formula,
			\[SSLF = SSE(R)-SSE(F)\]
			Substitute in the appropriate values to determine SSLF,
			\[SSLF = 548736108 - 5552112\]
			\[SSLF = 543183996\]
			Now, substitute in the values to compute F*,
			\[F* = \frac{543183996/1}{5552112/(84-2)}\]
			\[F* = \frac{543183996}{5552112/82}\]
			\[F* = \frac{543183996}{67708.68293}\]
			\[F* = 8022.368365\]
			\item Determine the decision from the F-test.
			Construct the following hypothesis test scenario,
			\[H_0: \beta_1 = 0\]
			\[H_a: \beta_1 \neq 0\]
			Now, $F* \thicksim F(df(SSLF), df(SSE(F))) $. 
			
			Find $df(SSLF)$ using the formula,
			\[df(SSLF) = df(SSE(R)) - df(SSE(F))\]
			Since SSE(F) = SSE, then from ANOVA Table,
			\[df(SSE(F)) = df(SSE) = 82\]
			Similarly, since SSE(R) = SSTO, from the ANOVA Table,
			\[df(SSE(R)) = df(SSTO) = 83\]
			Evaluate using the correct values, yielding,
			\[df(SSLF) = 83 - 82\]
			\[df(SSLF) = 1\]
			So, then we know,
			\[F* \thicksim F(1,82)\]
			We need to find the p-value, which can be computed like so,
			\[p-value = P(F > F*_{obs})\]
			\[p-value = P(F > 8022.368365)\]
			Using the following R code to calculate the p-value,
			\begin{lstlisting}
			p_value <- pf(8022.368365,1,82, lower.tail = FALSE)
			p_value
			\end{lstlisting}
			And so, the p-value for the F-test is,
			\[p-value = 1.42806 *10^{-83}\]
			We are testing using $\alpha = 0.01$, but regardless of whatever $\alpha$ we used, we see that,
			\[p-value = 1.42806 *10^{-83} < \alpha\]
			Therefore reject $H_0$ and conlcude $H_a$. Thus, there is reason to believe that crime rate and high school graduation rate a related.
		\end{enumerate}
		\item Finally, we need to determine if the decision rule from using F* is the same as using the T-test in 2.30 (a).
		\begin{enumerate}[label = \arabic*)]
			\item While both tests ended up rejecting $H_0$ because the p-value $< \alpha$, the two p-values were different. The p-values are as follows,
			\[p-value \quad for \quad T-test = 0.001202528\]
			\[p-value \quad for \quad F-test = 1.42806 *10^{-83}\]
			\item 	Even though these p-values will reach the same conclusion for most values of $\alpha$, they are not technically equivalent, and thus, the two tests are not truly equivalent either.
		\end{enumerate}
	\end{enumerate}
\end{document}