\documentclass{article}
\usepackage{amsfonts}
\usepackage{enumitem}
\usepackage{graphicx}
\usepackage{amssymb}

\title{STT 4660 \\ Homework \#3}
\author{Timothy Stubblefield}
\date{\today}

\usepackage[svgnames]{xcolor}
\usepackage{listings}

\lstset{language=R,
	basicstyle=\small\ttfamily,
	stringstyle=\color{DarkGreen},
	otherkeywords={0,1,2,3,4,5,6,7,8,9},
	morekeywords={TRUE,FALSE},
	deletekeywords={data,frame,length,as,character},
	keywordstyle=\color{blue},
	commentstyle=\color{DarkGreen},
}

\begin{document}
	
\maketitle

\section*{1.27}
	\begin{enumerate}[label = \alph*)]
		\item We need to find the regression function for the Muscle Mass dataset.
		\begin{enumerate}[label = \arabic*)]
			\item By using SAS, I obtained the regression function for Muscle Mass, which is
				\[\hat{y} = 156.34656 - 1.19000x\]
			\item The linear regression function seems to give a decent fit. 
				The $R^2 = 0.7501$ which means only $75$ percent of the points are explained by the model. In general, we would want the $R^2$ to be a little higher, but it isn't noticeably low either.
			\item Overall, the model supports the anticipation that muscle mass decreases with
			age since it shows that on average, for each year a person increases in age, their muscle mass measure decreases by about 1.21. 
			\item Below is the plot for the regression line from R,
				\centering
				\includegraphics[width = 10cm]{C:/Users/timst/R/Practice/STT 4660/Plot_for_1.27.png}
			\item Upon viewing the regression line for the data, we can visually see that the $R^2$ value is  a bit low and it seems that perhaps another model could fit the data better.
		\end{enumerate} 
		\item Now we want to solve the following problems.
		\begin{enumerate}[label = \arabic*)]
			\item Well, we want a point estimate for muscle mass of women differing by one year in age. From the regression model, we calculated that $b_1 = -1.19000$ which can be used to estimate $\beta_1$. Thus, if we were to find a point estimate of two women differing in age by one year, that could be represented as such, 
			\[W_1: y_{x} = 156.34656 - 1.19000x\]
			\[W_2: y_{x+1} = 156.34656 - 1.19000(x+1)\]
			Now, compute $W_2 - W_1$, yielding,
			\[W_2 - W_1 = 156-1.19000x-1.19000 - (156.34656-1.19000x)\]
			\[W_2 - W_1 = -1.19000\]
			Thus, on average, as a woman's age increases by 1 year, her muscle mass decreases by 1.19000 units.
			\item Now we want a point estimate for X = 60.
			Thus, we use the regression model, that is,
			\[\hat{y} = 156.34656 - 1.19000x\]
			So,
			\[\hat{y}_{60} = 156.34656 - 1.19000(60)\]
			\[\hat{y}_{60} = 156.34656 - 71.4\]
			\[\hat{y}_{60} = 84.94656\]
			\item Now, we want to find the residual for the eighth case. 
			For the eighth case, we have the following data,
			\[X_8 = 41 \qquad \qquad Y_8 = 112\]
			So, $Y_8 = 112$ occurs when $x = 41$, thus that means,
			\[y_{8} = y(41) = 112\]
			Now, use the regression model to find $\hat{y}_{8}$, as such,
			\[\hat{y}_{8} = 156.34656 - 1.19000(41)\]
			\[\hat{y}_{8} = 156.34656 - 44.69\]
			\[\hat{y}_{8} = 111.65656\]
			Hence, the eighth residual will be $e_8 = y_{8}-\hat{y}_{8}$.
			Evaluate $e_8$ as such,
			\[e_8 = 112 - 111.65656\]
			\[e_8 = 0.34344\]
			\item Finally, let's find a point estimate for $\sigma^2$.
			To find a point estimate for $\sigma^2$, we will use MSE.
			The formula for MSE is,
			\[MSE = \hat{\sigma}^2 = \frac{SSE}{n-2}\]
			Previously, we used SAS to generate the regression model. Well, SAS also computed MSE along with other metrics.
			Thus, our point estimate for $\sigma^2$ is,
			\[MSE = \hat{\sigma}^2 = 66.80082\]			
		\end{enumerate}
	\end{enumerate}

\section*{2.2}
	\begin{enumerate}
		\item Well, in the hypothesis test, the analyst concluded $H_0: \beta_1 \leq 0$ for the regression model $\hat{y} = \beta_0 + \beta_1x$
		\item In order for their to be no linear relationship between X and Y, the analyst would have to conclude that $\beta_1 = 0$
		\item If $\beta_1 = 0$, that reduces the linear regression function to 
		\[\hat{y} = \beta_0\]
		\item However, since the analyst concluded $H_0: \beta_1 \leq 0$, that 
		means that either $\beta_1 < 0$ or $\beta_1 = 0$.
		\item Therefore, if $\beta_1 = 0$, then there would be no relationship between X and Y, but if $\beta_1 < 0$ then X and Y have a negative relationship.
		\item Thus, with the concluded hypothesis, the relationship between X and Y is inconclusive since it could be negative or none depending on the value of $\beta_1$.
	\end{enumerate}

\section*{2.3}
	\begin{enumerate}
		\item So the student's estimated regression function is,
		\[\hat{Y} = 350.7 -0.18X\]
		\item However, the student was studying the relationship between advertising costs(X) and sales(Y), which means we would not expect a negative relationship since usually the more money you spend on advertising, the more sales you have. 
		\item We also know that the Two sided p-value for the estimated slope is 0.91.
		\item Based on the results for the regression line, the hypothesis test must have been,
		\[H_0: \beta_1 = 0\]
		\[H_a: \beta_1 = -0.18\]
		\item Now, since p-value = 0.91, that means that,
		\[p-value > \alpha\] since we would never choose an $\alpha > 0.91$ because that would be a 9 percent confidence interval or less, which is useless.
		\item Thus, the student should not reject $H_0$, however they incorrectly rejected $H_0$ and concluded $H_a$. As a result, the student is claiming that more advertising reduces sales, which is not a correct conclusion.
	\end{enumerate}

\section*{2.6}
	\begin{enumerate}[label = \alph*)]
		\item We want to estimate $\beta_1$ with a 95 percent confidence interval for Airfreight Breakage.
		\begin{enumerate}[label = \arabic*)]
			\item Using SAS (and from Problem 1.21), I generated the regression function for Airfreight Breakage.
			The regression function is,
			\[\hat{y} = 10.2 +4x\]
			So, $b_0 = 10.2$ and $b_1 = 4$
			\item From the notes, we know that the 100(1-$\alpha$) confidence interval for $b_1$ is,
			\[b_1 \pm t_{\alpha/2}(n-2)* S_{b_1}\]
			\item However, first we must calculate $S_{b_1}$ with the following formula,
			\[S_{b_1}^2 = \frac{\hat{\sigma}^2}{S_{xx}}\] where $\hat{\sigma}^2 = MSE$
			\item From Problem 1.21, we have the the following results,
			\[MSE = \hat{\sigma}^2 = 2.2\]
			\[S_{xx} = \sum_{i=1}^n x_i^2 - n\bar{x}^2 = 10\]
			\item Now, we can use the results from (4) to find $S_{b_1}$ like so,
			\[S_{b_1}^2 = \frac{2.2}{10}\]
			\[S_{b_1}^2 = 0.22\]
			\item And, now we can compute $S_{b_1}$ as such,
			\[S_{b_1} = \sqrt{S_{b_1}^2} = \sqrt{0.22}\]
			\[S_{b_1} = 0.469041576\]
			\item Now, we need to calculate $t_{\alpha/2}(n-2)$
			\item Since we want the 95 percent Confidence Interval, so
			\[\alpha = 0.05 \qquad \qquad \frac{\alpha}{2} = 0.025\ \qquad \qquad n=10\]
			\item Now, use $\alpha$ to find the t-value (in a two sided test),
			\[t_{\alpha/2}(n-2) = t_{0.025}(10-2)\]
			\[t_{\alpha/2}(n-2) = t_{0.025}(8)\]
			\item Using R, I calcuated the t-value,
			\[t_{0.025}(8) = 2.306004\]
			\item So, substitute in values to the formula find the confidence interval,
			\[b_1 \pm t_{\alpha/2}(n-2)*S_{b_1}\]
			\[4 \pm t_{0.025}(8)*S_{b_1}\]
			\[4 \pm 2.306004*0.469041576\]
			\[4 \pm 1.08161175\]
			\item Thus, the 95 percent confidence interval for $\beta_1$ is,
			\[(2.91838825, 5.08161175)\]
			\item Therefore, we a 95 percent confident that $\beta_1$ lies in the range of (2.91838825, 5.08161175).
		\end{enumerate}
		\item Now, we want to do a t-test along with hypothesis testing to determine if there is a relationship between X and Y.
		\begin{enumerate}[label = \arabic*)]
			\item So, first since we want to test whether or not X and Y are related, our hypothesis test situation will be,
			\[H_0: \beta_1 = 0\]
			\[H_a: \beta_1 \neq 0\]
			\item If we conclude $H_0$, then there is evidence to show that there is not likely a relationship between X and Y, but it we conclude $H_a$, then the evidence shows that there is likely a relationship between X and Y.
			\item Now, we calculate the t-value with the formula,
			\[t = \frac{b_1 - \beta_{10}}{S_{b_1}}\]
			\item From Problem 2.6 (b), we know that,
			\[b_1 = 4, \qquad S_{b_1} = 0.469041576\]
			And, we are testing against to see if $\beta_1 =0$, so $\beta_{10} = 0$
			\item Substitute in the above values to calculate t-value,
			\[t_{obs} = \frac{4-0}{0.469041576}\]
			\[t_{obs} = \frac{4}{0.469041576}\]
			\[t_{obs} = 8.528028654\]
			\item From R, I computed the p-value using the following code,
			\begin{lstlisting}
			t_obs <- 8.528028654
			#df <- n-1
			df <- 9
			2*pt(-abs(t), df=9)
			\end{lstlisting}
			\item The computed p-value is 0.04653987
			\item Compare the p-value to $\alpha$,
			\[p-value = 0.04653987 < 0.05 = \alpha\]
			Thus, reject $H_0$ and conclude $H_a$
			\item Hence, there is strong evidence to suggest that there is a relationship between X and Y.
		\end{enumerate}
		\item Finally, we want to find a 95 percent confidence interval for $\beta_0$ and interpret it.
		\begin{enumerate}[label = \arabic*)]
			\item From the notes, we have that a $100(1-\alpha)$ percent confidence level for $\beta_0$ is,
			\[b_0 \pm t_{\alpha/2}(n-2)*S_{b_0}\]
			\item Now, we need to compute $S_{b_0}$, so we use the formula,
			\[S_{b_0}^2 = \hat{\sigma}^2(\frac{1}{n}+\frac{\bar{x}^2}{S_{xx}})\]
			\item Recall that $\hat{\sigma}^2 = MSE = \frac{SSE}{n-2}$ from the notes.
			\item From Problem 1.25 (b) in Homework 2, I calculated  the MSE, that is,
			\[\hat{\sigma}^2 = MSE = 2.2\]
			\item From the same Homework, I calculated $S_{xx}$, that is,
			\[S_{xx} = \sum_{i=1}^n x_i^2 - n\bar{x}^2 = 10\]
			And n = 10, $\bar{x} = 1$, and $b_0 = 10.2$
			\item Now, we just substitute each of these values into the formula for $S_{b_0}^2$, yielding,
			\[S_{b_0}^2 = 2.2(\frac{1}{10}+\frac{1^2}{10})\]
			\[S_{b_0}^2 = 2.2(\frac{1}{10}+\frac{1}{10})\]
			\[S_{b_0}^2 = 2.2(\frac{1}{5})\]
			\[S_{b_0}^2 = 0.44\]
			\item Now, take the square root of $S_{b_0}^2$ and get,
			\[S_{b_0} = \sqrt{0.44} = 0.6633249581\]
			\item Since we are using a 95 percent confidence interval, with n = 10, we have the same t-value as the confidence interval from 2.6 (a),
			\[t_{\alpha/2}(n-2) = t_{0.025}(8) = 2.306004\]
			\item Finally, we have the necessary values to compute the confidence interval of $\beta_0$, as such,
			\[b_0 \pm t_{\alpha/2}(n-2)*S_{b_0}\]
			\[10.2 \pm 2.306004*0.6633249581\]
			\[10.2 \pm 1.529630007\]
			\item Thus, the 95 percent confidence interval for $\beta_0$ is,
			\[(8.670369993, 11.72963001)\]
			\item Therefore, we are 95 percent confident that $\beta_0$ lie in the range of (8.670369993, 11.72963001).
		\end{enumerate}
	
	\end{enumerate}
\end{document}